\documentclass[compress]{beamer}

\usepackage[utf8]{inputenc}
\usepackage[french]{babel}
\usetheme{Amsterdam}

\usepackage{multicol}
\usepackage{listings}
\usepackage{color}

\usepackage{tikz}
\usetikzlibrary{shapes,arrows}

\lstset{language=caml, frame=single, basicstyle=\ttfamily\scriptsize, commentstyle=\color{dkgreen}}

\newcommand{\email}[0]{francois.ripault@epita.fr}

\definecolor{dkgreen}{rgb}{0,0.6,0}

\defbeamertemplate*{title page}{customized}[1][]
{
  \begin{center}
  \usebeamerfont{title}\inserttitle\par
    \usebeamerfont{subtitle}\usebeamercolor[fg]{subtitle}\insertsubtitle\par
    \bigskip
    \usebeamerfont{author}\insertauthor\par
    \scriptsize \texttt{\email} \par
    \usebeamerfont{institute}\insertinstitute\par
    \vfill
    \usebeamerfont{date}\insertdate\par
    %\usebeamercolor[fg]{titlegraphic}\inserttitlegraphic
    \end{center} }

\newenvironment{tframe}[1]{
  \subsection{#1}
  \begin{frame}{#1}
  }{
  \end{frame}
  }

%% FIXME: title
\title{Introducing the new Coqdoc}

\author{François Ripault}

\begin{document}
\begin{frame}
\titlepage
\end{frame}

\begin{frame}{Introduction}
  \begin{columns}[2]
    \begin{column}{0.5\textwidth}
      What is Coqdoc ?
      \begin{itemize}
        \item Documentation tool for Coq
        \item Many possible use cases
      \end{itemize}
    \end{column}
    \begin{column}{0.5\textwidth}
      Why a new Coqdoc ?
      \begin{itemize}
        \item Coqdoc is hard to maintain
        \item Lack of extensibility
        \item Better integration of the tool
      \end{itemize}
    \end{column}
  \end{columns}
\end{frame}

\begin{frame}
    \tableofcontents
\end{frame}

\section{Coqdoc-ng's architecture}
  \begin{frame}{Coqdoc-ng's architecture}
    %% FIME : pointe flèches
    \begin{figure}
  \begin{tikzpicture}

  \node[draw, ellipse, thick, fill=blue!20] (V) at (-3,0) {Coq File};
  \node[draw, ellipse, thick, fill=blue!20] (Tex) at (-3,-1) {Coq-tex File};

  \node[draw, thick, fill=blue!20] (front) at (1,-0.5) {Front-end};
  \draw[->] (V.east) -- (front.west);
  \draw[->] (Tex.east) -- (front.west);

  \node[draw, thick, fill=blue!20] (eval) at (1, -2) {Evaluation};
  \draw[->, -latex'] (front.south) -- node[right, color=black] {AST} (eval.north);

  \node[draw, thick, fill=red!20] (coqtop) at (5,-2) {Coqtop};
  %\draw[->, dashed, -latex'] (eval.east)  -- node[above, color=black] {Interaction} (coqtop.west);
  \draw[<->, dashed] (eval.east)  -- node[above] {Interaction} (coqtop.west);

  \node[draw, thick, fill=blue!20] (back) at (1, -4) {Back-end};
  \draw[->] (eval.south) -- (back.north);

  \node[draw, thick, fill=yellow!20] (spec-html) at (-3, -3.5) {Html spec};
  \draw[->,-latex'] (spec-html.east) -- node[above] {Specification} (back.west);
  \node[draw, thick, fill=yellow!20] (spec-latex) at (-3, -4.5) {LaTeX spec};
  \draw[->] (spec-latex.east) -- (back.west);

  \node[draw, thick, ellipse, fill=blue!20] (output) at (5, -4.5) {Output file(s)};
  \draw[->,-latex'] (back.east) -- node[above] {Print} (output.west);

  \end{tikzpicture}
\end{figure}

  \end{frame}
  \begin{frame}[containsverbatim]{Coqdoc's input files}
    \lstinputlisting{input.v}
    \end{frame}

  \begin{tframe}{Front-end}
    Role :
    \begin{itemize}
      \item Translation into an abstract representation
      \item Multiple parsers and lexers
      \item Selected parser depends on the input file/option
    \end{itemize}

    \vfill
    Two processing phases :
    \begin{enumerate}
      \item Separation between comments, documentation and code
      \item Documentation processing
    \end{enumerate}
  \end{tframe}
  \begin{frame}{Front-end}
      Abstract representation  :
      \begin{itemize}
        \item Comments : \texttt{string}
        \item Code : \texttt{string} (for now $\ldots$)
        \item Documentation : \\
          \begin{itemize}
            \item Simple elements (content, title, horizontal rule)
            \item Recursive elements (lists, emphasis)
            \item Elements subject to evaluation (query, printing rule)
          \end{itemize}
      \end{itemize}
    \end{frame}
    \begin{frame}[containsverbatim]{Our document}
    \begin{center}
      \scalebox{0.7}{
      \begin{tikzpicture}
      \node[draw, thick, fill=blue!20] (1) at (0,0) {Doc :       " * Example\_doc $\ldots$"};
      \node[draw, thick, fill=blue!20] (2) at (0, -2) {Comment : " Now we write $\ldots$"};
      \node[draw, thick, fill=blue!20] (3) at (0,-4) {Code :     "Definition id $\ldots$"};
      \node[draw, thick, fill=blue!20] (4) at (0,-6) {Comment : "This returns the $\ldots$"};
      \node[draw, thick, fill=blue!20] (5) at (0,-8) {Code : "Eval compute in $\ldots$"};

      \draw[->] (1.south) -- (2.north);
      \draw[->] (2.south) -- (3.north);
      \draw[->] (3.south) -- (4.north);
      \draw[->] (4.south) -- (5.north);
    \end{tikzpicture}
  }
    \end{center}
    \end{frame}
    \begin{frame}{After parsing the documentation}
    \scalebox{0.7}{
    \begin{tikzpicture}
      \node[draw, thick, fill=blue!20] (1) at (-3,0) {Doc : Title};
      \node[draw, thick, fill=blue!20] (1a) at (0,0) {Doc : Content};
      \node[draw, thick, fill=blue!20] (1b) at (3,0) {Doc : Hrule};
      \node[draw, thick, fill=blue!20] (1c) at (6,0) {Doc : List};
      \node[draw, thick, fill=blue!20] (1d) at (4.5,-2) {Doc : Item};
      \node[draw, thick, fill=blue!20] (1e) at (7.5,-2) {Doc : Item};
      \node[draw, thick, fill=blue!20] (2) at (3, -4) {Comment : " Now we write $\ldots$"};
      \node[draw, thick, fill=blue!20] (3) at (3,-5) {Code :     "Definition id $\ldots$"};
      \node[draw, thick, fill=blue!20] (4) at (3,-6) {Comment : "This returns the $\ldots$"};
      \node[draw, thick, fill=blue!20] (5) at (3,-7) {Code : "Eval compute in $\ldots$"};

      \draw[->] (1.east) -- (1a.west);
      \draw[->] (1a.east) -- (1b.west);
      \draw[->] (1b.east) -- (1c.west);
      \draw[->] (1c.south) -- (1d.north);
      \draw[->] (1c.south) -- (1e.north);

      \draw[->] (1c.east) to[out=0, in=90] (9,-2) to[out=-90,in=90] (2.north);
      \draw[->] (2.south) -- (3.north);
      \draw[->] (3.south) -- (4.north);
      \draw[->] (4.south) -- (5.north);
    \end{tikzpicture}
  }
    \end{frame}


    \subsection{Evaluation}
    \begin{frame}[containsverbatim]{Evaluation}
    Translation for each type :
    \begin{itemize}
      \item Comments : \\
         \begin{itemize}
           \item Treat show commands : set a global state
             \begin{lstlisting}
(* begin hide *)
  some code ...
(* end hide*)
\end{lstlisting}
           \item Remove every comment
        \end{itemize}
      \item Documentation : Translate elements subject to evaluation \\
        \begin{itemize}
          \item Query : apply the function
\begin{lstlisting}
(** @query{some,arguments} *)
\end{lstlisting}

          \item Printing rule : set a new printing rule in the global state
          \item Apply the printing rules on the rest
        \end{itemize}
      \item Code : interaction with Coqtop
    \end{itemize}
  \end{frame}
  \begin{frame}[After the evaluation]
    \scalebox{0.7}{
    \begin{tikzpicture}
      \node[draw, thick, fill=blue!20] (1) at (-3,0) {Doc : Title};
      \node[draw, thick, fill=blue!20] (1a) at (0,0) {Doc : Content};
      \node[draw, thick, fill=blue!20] (1b) at (3,0) {Doc : Hrule};
      \node[draw, thick, fill=blue!20] (1c) at (6,0) {Doc : List};
      \node[draw, thick, fill=blue!20] (1d) at (5,-2) {Doc : Item};
      \node[draw, thick, fill=blue!20] (1e) at (8,-2) {Doc : Item};
      \node[draw, thick, fill=blue!20] (3) at (2,-4) {Code :     "Definition id $\ldots$"};
      \node[draw, thick, fill=blue!20] (5) at (2,-6) {Code : "Eval compute in $\ldots$"};
      \draw[->] (1.east) -- (1a.west);
      \draw[->] (1a.east) -- (1b.west);
      \draw[->] (1b.east) -- (1c.west);
      \draw[->] (1c.south) -- (1d.north);
      \draw[->] (1c.south) -- (1e.north);
      \draw[->] (1c.east) to[out=0, in=90] (10,-2) to[out=-90,in=90] (3.north);
      \draw[->] (3.south) -- (5.north);
    \end{tikzpicture}
  }
  \end{frame}


  \begin{tframe}{Interaction with Coqtop}
    Why interact with Coqtop ?
    \begin{itemize}
      \item Remove code processing in Coqdoc
      \item Xml protocol (already used by CoqIDE)
      \item Easy to add new mechanisms
    \end{itemize}
    \vfill
    Objectives for code processing :
    \begin{itemize}
      \item Indentation
      \item Syntactic coloration
      \item Identifier processing
      \item Notations processing %%FIXME
    \end{itemize}
  \end{tframe}

  \begin{tframe}{New commands : \texttt{locate} and \texttt{prettyprint}}
    \begin{itemize}
      \item \texttt{locate} : locates an identifier
        \begin{itemize}
          \item From a name, returns its ``absolute'' name
          \item Generate hyperlinks between different modules being documented
          \item Easy to implement
        \end{itemize}
        \vfill
      \item \texttt{prettyprint} : indents and annotates code
        \begin{itemize}
          \item Coqtop indents and tags each element
          \item Coqdoc translates the tags into final elements \\
            \small (syntactic coloration, processing of identifiers, $\ldots$)
          \item Hard to do : no good level of abstraction \\
            \small (lack of a concrete syntax tree)
        \end{itemize}
    \end{itemize}
  \end{tframe}

  \begin{tframe}{An imperfect solution for Prettyprint}

    Method :
    \begin{itemize}
      \item Use Coq's \texttt{Printing} module
      \item Annotate the output with tags
      \item Semantic kept and indentation obtained
    \end{itemize}
    Implementation :
    \begin{itemize}
      \item Make tags based on AST types (VernacExpr and ConstrExpr)
      \item Each node translation is surrounded with a tag function
      \item Generates Xml content carrying the tags
    \end{itemize}
  \end{tframe}

  \begin{frame}[containsverbatim]{Code modification example}

    Modification of the printing module (in red) :
    \begin{lstlisting}[escapeinside=\<\>]
(* some code *)
| CApp (_,(None,a),l) ->
  <\textcolor{red}{tag C\_CApp}> (pr_app (pr mt) a l), lapp
(* some code *)
    \end{lstlisting}

    The \texttt{tag} function surrounds the output with XML tags \\
    Example output :
    \begin{lstlisting}[language=xml]
<checkmayeval>Eval compute in
  <cnotation>
    <unpmetavar> <prim>5</prim></unpmetavar>
    <unpterminal> +</unpterminal>
    <unpmetavar><prim>5</prim></unpmetavar>
  </cnotation>
</checkmayeval>.
\end{lstlisting}
  \end{frame}

  \begin{tframe}{Processing rules for the code}
        Main tasks :
        \begin{itemize}
          \item Annotate code for syntactic coloration
          \item Apply printing rules
          \item Process identifiers
        \end{itemize}
        \vfill
        Chain of responsibility design pattern :
        %% FIXME ?
        \begin{itemize}
          \item Apply a set of commands to a data
          \item Each command is a function taking the next command to apply as an argument
        \end{itemize}
        Advantages :
        \begin{itemize}
          \item Easy to add rules
          \item Extensible
          \item Possibility to chain commands
        \end{itemize}
        Output : better type for representing code
  \end{tframe}

  \begin{frame}{Code type}
    Three main code types :
    \begin{itemize}
      \item Formatting : keywords, literals, tactics, $\ldots$
      \item Identifiers : contain hyperlinks
      \item No format
    \end{itemize}
  \end{frame}

  \begin{tframe}{Identifiers}
    Identifier processing chain :
    \begin{itemize}
      \item Use \texttt{locate} command on each identifier
      \item Ignore undeclared identifiers
      \item Special cases for standard library or external library
      \item Generate a link type used in the output module.
    \end{itemize}
  \end{tframe}
  \begin{frame}{Code and hyperlinks Result}
    \scalebox{0.7}{
    \begin{tikzpicture}
      \node[draw, thick, fill=blue!20] (1) at (-3,0) {Doc : Title};
      \node[draw, thick, fill=blue!20] (1a) at (0,0) {Doc : Content};
      \node[draw, thick, fill=blue!20] (1b) at (3,0) {Doc : Hrule};
      \node[draw, thick, fill=blue!20] (1c) at (6,0) {Doc : List};
      \node[draw, thick, fill=blue!20] (1d) at (5,-2) {Doc : Item};
      \node[draw, thick, fill=blue!20] (1e) at (8,-2) {Doc : Item};

      \node[draw, thick, fill=blue!20] (3) at (-3,-5) {Code : Keyword \textcolor{red}{"Definition"}};
      \node[draw, thick, fill=blue!20] (3a) at (2,-5) {Code : Id      \textcolor{red}{"id"}};
      \node[draw, thick, fill=blue!20, minimum size=0.7cm] (3b) at (5,-5) {$\ldots$};

      \node[draw, thick, fill=blue!20] (5) at (-3,-8) {Code : Keyword \textcolor{red}{"Eval"}};
      \node[draw, thick, fill=blue!20, minimum size=0.7cm] (5a) at (1,-8) {$\ldots$};
      \node[draw, thick, fill=blue!20] (5b) at (4,-8) {Code : Id \textcolor{red}{"id"}};
      \node[draw, thick, fill=blue!20] (5c) at (8,-8) {Code : Literal \textcolor{red}{"42"}};

      \draw[->] (1.east) -- (1a.west);
      \draw[->] (1a.east) -- (1b.west);
      \draw[->] (1b.east) -- (1c.west);
      \draw[->] (1c.south) -- (1d.north);
      \draw[->] (1c.south) -- (1e.north);
      \draw[->] (1c.east) to[out=0, in=90] (10,-1) to[out=-90,in=90] (3.north);

      \draw[->] (3.east) -- (3a.west);
      \draw[->] (3a.east) -- (3b.west);

      \draw[->] (5.east) -- (5a.west);
      \draw[->] (5a.east) -- (5b.west);
      \draw[->] (5b.east) -- (5c.west);
      \draw[->] (3b.south) to[out=-90, in=90] (5.north);

      \draw[->,dashed, color=blue] (3a.south) -- (5b.north);
    \end{tikzpicture}
  }
  \end{frame}

  \begin{tframe}{Back-end}
    Back-end structure
    \begin{itemize}
      \item Central module in charge of printing
      \item Specification modules for each output type
        \begin{itemize}
          \item Printing functions for each type (documentation and code)
          \item Hyperlink generation function
          \item Index generation functions
        \end{itemize}
    \end{itemize}
  \end{tframe}

\section{Extending Coqdoc}
\subsection{Extending the notations}
\begin{frame}[containsverbatim]{Extending the notations}
  A richer notations system :
  \begin{itemize}
    \item Generate output specific commands for the code
      the code : \texttt{sum(a,b)} becomes
      \begin{verbatim}
      \sum^{a}_{b}
      \end{verbatim}
  \end{itemize}
\end{frame}
\begin{tframe}{New file format: Vdoc}
  A new file format : \texttt{.vdoc}
  \begin{itemize}
    \item Portable file : back-end independent
    \item Saved after evaluation
    \item Used to generate Coqdoc's documents
    \item Better control over generated files
  \end{itemize}
\end{tframe}

  \begin{tframe}{Coq-tex}
    How coq-tex works :
    \begin{itemize}
      \item Code in 3 environments :
        \begin{itemize}
          \item \texttt{coq\_example} : print input and result of evaluation
          \item \texttt{coq\_example*} : only print input
          \item \texttt{coq\_eval} : only print the result of evaluation
        \end{itemize}
      \item Everything else is left unprocessed
    \end{itemize}
    Parts to add :
    \begin{itemize}
      \item Add a new front-end : each environment is translated into a
        query type
      \item Process queries
      \item Modify the $\LaTeX$ back-end
      \item Add Coq-tex options to Coqdoc
    \end{itemize}
  \end{tframe}

\section{Demo}
  \begin{frame}{Demo}
    \begin{center}
      Demo
    \end{center}
  \end{frame}

  \section{Remaining work}
  \begin{frame}{Remaining work}
    What's left to do :
    \begin{itemize}
      \item Bugfixes
      \item Retro-compatibility (mostly options)
      \item Clean errors
    \end{itemize}
    Two scopes of improvement :
    \begin{itemize}
      \item XML protocol
      \item Coqdoc
    \end{itemize}
  \end{frame}

  \begin{frame}{Conclusion}
    State of Coqdoc :
    \begin{itemize}
      \item Mostly working prototype
      \item Little work left to do before release
      \item Space for enrichment %FIXME ?
    \end{itemize}
    \vfill
  \begin{center} \large Questions ? \end{center}
  \end{frame}

\end{document}
