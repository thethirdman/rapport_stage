\documentclass[compress]{beamer}

\usepackage[utf8]{inputenc}
\usepackage[french]{babel}
\usetheme{Amsterdam}

\newcommand{\email}[0]{francois.ripault@epita.fr}

\defbeamertemplate*{title page}{customized}[1][]
{
  \begin{center}
  \usebeamerfont{title}\inserttitle\par
    \usebeamerfont{subtitle}\usebeamercolor[fg]{subtitle}\insertsubtitle\par
    \bigskip
    \usebeamerfont{author}\insertauthor\par
    \scriptsize \texttt{\email} \par
    \usebeamerfont{institute}\insertinstitute\par
    \vfill
    \usebeamerfont{date}\insertdate\par
    %\usebeamercolor[fg]{titlegraphic}\inserttitlegraphic
    \end{center}
}

\newenvironment{tframe}[1]{
  \subsection{#1}
  \begin{frame}{#1}
  }{
  \end{frame}
  }

%% FIXME: title
\title{Presenting the new Coqdoc}

\author{François Ripault}


\begin{document}
\begin{frame}
\titlepage
\end{frame}

\section{Introduction}
  \begin{tframe}{What is Coqdoc ?}
  \begin{itemize}[<+->]
  \end{itemize}
  \end{tframe}

  \begin{tframe}{Why a new Coqdoc ?}
  \end{tframe}

\section{Coqdoc-ng's architecture}
  \begin{frame}{Coqdoc-ng's architecture}
  \end{frame}

  \begin{tframe}{Frontend}
  \end{tframe}

  \begin{tframe}{Evaluation}
  \end{tframe}


  \begin{tframe}{Interaction with Coqtop}
  \end{tframe}

  \begin{tframe}{New commands : Locate and Prettyprint}
  \end{tframe}

  \begin{tframe}{An imperfect solution for Prettyprint}
  \end{tframe}

  \begin{tframe}{Backend}
  \end{tframe}

\section{Demo}
  \begin{tframe}{Demo}
  \end{tframe}

\section{Remaining work}
  \begin{tframe}{What's left to do}
  \end{tframe}
  \begin{tframe}{New possibilities for Coqdoc}
  \end{tframe}

\section{Conclusion}
  \begin{tframe}{Conclusion}
  \end{tframe}

\end{document}
