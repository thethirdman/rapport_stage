\documentclass[compress]{beamer}

\usepackage[utf8]{inputenc}
\usepackage[french]{babel}
\usetheme{Amsterdam}
\usepackage{tikz}
\usetikzlibrary{shapes,arrows}

\newcommand{\email}[0]{francois.ripault@epita.fr}

\defbeamertemplate*{title page}{customized}[1][]
{
  \begin{center}
  \usebeamerfont{title}\inserttitle\par
    \usebeamerfont{subtitle}\usebeamercolor[fg]{subtitle}\insertsubtitle\par
    \bigskip
    \usebeamerfont{author}\insertauthor\par
    \scriptsize \texttt{\email} \par
    \usebeamerfont{institute}\insertinstitute\par
    \vfill
    \usebeamerfont{date}\insertdate\par
    %\usebeamercolor[fg]{titlegraphic}\inserttitlegraphic
    \end{center}
}

\newenvironment{tframe}[1]{
  \subsection{#1}
  \begin{frame}{#1}
  }{
  \end{frame}
  }

%% FIXME: title
\title{Presenting the new Coqdoc}

\author{François Ripault}


\begin{document}
\begin{frame}
\titlepage
\end{frame}

\begin{frame}{Introduction}
  \begin{columns}[2]

    \begin{column}{0.5\textwidth}
      What is Coqdoc ?
      \begin{itemize}
        \item Documentation tool for Coq
        \item Many possible use cases
        \item We would like to do more with it
      \end{itemize}
    \end{column}

    \begin{column}{0.5\textwidth}
      Why a new Coqdoc ?
      \begin{itemize}[<+->]
        \item Coqdoc is hard to maintain
        \item Lack of extensibility
        \item Better integration of the tool
      \end{itemize}
    \end{column}

  \end{columns}
\end{frame}

\begin{frame}
  \tableofcontents
\end{frame}

\section{Coqdoc-ng's architecture}
  \begin{frame}{Coqdoc-ng's architecture}
    %% FIME : pointe flèches
    \begin{figure}
  \begin{tikzpicture}

  \node[draw, ellipse, thick, fill=blue!20] (V) at (-3,0) {Coq File};
  \node[draw, ellipse, thick, fill=blue!20] (Tex) at (-3,-1) {Coq-tex File};

  \node[draw, thick, fill=blue!20] (front) at (1,-0.5) {Front-end};
  \draw[->] (V.east) -- (front.west);
  \draw[->] (Tex.east) -- (front.west);

  \node[draw, thick, fill=blue!20] (eval) at (1, -2) {Evaluation};
  \draw[->, -latex'] (front.south) -- node[right, color=black] {AST} (eval.north);

  \node[draw, thick, fill=red!20] (coqtop) at (5,-2) {Coqtop};
  %\draw[->, dashed, -latex'] (eval.east)  -- node[above, color=black] {Interaction} (coqtop.west);
  \draw[<->, dashed] (eval.east)  -- node[above] {Interaction} (coqtop.west);

  \node[draw, thick, fill=blue!20] (back) at (1, -4) {Back-end};
  \draw[->] (eval.south) -- (back.north);

  \node[draw, thick, fill=yellow!20] (spec-html) at (-3, -3.5) {Html spec};
  \draw[->,-latex'] (spec-html.east) -- node[above] {Specification} (back.west);
  \node[draw, thick, fill=yellow!20] (spec-latex) at (-3, -4.5) {LaTeX spec};
  \draw[->] (spec-latex.east) -- (back.west);

  \node[draw, thick, ellipse, fill=blue!20] (output) at (5, -4.5) {Output file(s)};
  \draw[->,-latex'] (back.east) -- node[above] {Print} (output.west);

  \end{tikzpicture}
\end{figure}

  \end{frame}

  \begin{tframe}{Front-end}
    Role :
    \begin{itemize}
      \item Translation into an abstract representation
      \item Multiple parsers and lexers
      \item Selected parser depends on the input file/option
    \end{itemize}

    \vfill
    Two processing phases :
    \begin{enumerate}
      \item Separation between comments, documentation and code
      \item Documentation processing
    \end{enumerate}
  \end{tframe}
  \begin{frame}{Front-end}
      Abstract representation  :
      \begin{itemize}
        \item Comments : \texttt{string}
        \item Code : \texttt{string} (for now $\ldots$)
        \item Documentation : \\
          \begin{itemize}
            \item Simple elements (content, title, horizontal rule)
            \item Recursive elements (lists, emphasis)
            \item Elements subject to evaluation (query, printing rule)
          \end{itemize}
      \end{itemize}
    \end{frame}

  \begin{tframe}{Evaluation}
    Translation for each type :
    \begin{itemize}
      \item Comments : \\
         \begin{itemize}
           \item Treat begin/end hide/show : set a global state
           \item Remove everything
        \end{itemize}
      \item Documentation : Translate elements subject to evaluation \\
        \begin{itemize}
          \item Query : apply the function
          \item Printing rule : set a new printing rule in the global state
          \item Apply the printing rules on the rest
        \end{itemize}
      \item Code : interaction with Coqtop
    \end{itemize}
  \end{tframe}


  \begin{tframe}{Interaction with Coqtop}
    Why interact with Coqtop ?
    \begin{itemize}[<+->]
      \item Xml protocol (already used by CoqIDE)
      \item Remove code processing in Coqdoc
      \item Easy to add new mechanisms
    \end{itemize}
    \vfill
    Objectives for code processing :
    \begin{itemize}
      \item Indentation
      \item Syntactic coloration
      \item Identifier processing
      \item Notations processing %%FIXME
    \end{itemize}
  \end{tframe}

  \begin{tframe}{New commands : Locate and Prettyprint}
    \begin{itemize}
      \item Locate : locate an identifier
        \begin{itemize}
          \item Generate hyperlinks
          \item Easy to do
        \end{itemize}
        \vfill
      \item Prettyprint : indents and annotates code
        \begin{itemize}
          \item Coqtop indents and tags each element
          \item Coqdoc translates the tags
          \item Hard to do : no CST
        \end{itemize}
    \end{itemize}
  \end{tframe}

  \begin{tframe}{An imperfect solution for Prettyprint}
    \begin{itemize}
      \item Use Coq's \texttt{Printing} module
      \item Annotate the output with tags
      \item Semantic and indentation obtained
    \end{itemize}
  \end{tframe}
  %FIXME
  \begin{frame}{Code processing diagram}
    \resizebox{\textwidth}{!}{\begin{figure}
  \begin{tikzpicture}
    %% Rectangle
    \draw (-2,-3) -- (-2,1);
    \draw (-2,-3) -- (5,-3);
    \draw (5,1) -- (-2,1);
    \draw (5,1) -- (5,-3);

    %% graphe1
    \node[text width=6cm] (titre1) at (3,1.5) {Arbre de syntaxe};
    \node[draw] (Noeud1) at (3,0) {Noeud1};
    \node[draw] (Expression1) at (1,-1) {Expresion1};
    \node[draw] (Element3) at (4,-1) {Element3};
    \node[draw] (Element1) at (0,-2) {Element1};
    \node[draw] (Element2) at (2,-2) {Element2};
    \draw[->] (Noeud1) -> (Expression1);
    \draw[->] (Noeud1) -> (Element3);
    \draw[->] (Expression1) -> (Element1);
    \draw[->] (Expression1) -> (Element2);

    %%print par défaut
    \draw (-5.5,-5) -- (-0.5,-5);
    \draw (-5.5,-5) -- (-5.5,-7);
    \draw (-5.5,-7) -- (-0.5,-7);
    \draw(-0.5,-5) -- (-0.5,-7);

    \node[text width=6cm] (titre3) at (-1,-4.25) {Sortie Texte};
    \node[text width=6cm] (Pp) at (-2,-6) {"Noeud1: \\ (Element1 + Element2):\\     ~~~~Element3"};


    \draw (0.8,-5) -- (0.8,-9) -- (9,-9) -- (9,-5) -- (0.8,-5);
    \node[text width=6cm] (titre3) at (5,-4.25) {Sortie Annotée};
    \node[text width=8cm] (Pp) at (5,-7) {
    "$\textless$noeud1$\textgreater$Noeud1: \\
     ~~~~$\textless$expression1$\textgreater$ \\
     ~~~~~~~~($\textless$element1$\textgreater$Element1$\textless$/element1$\textgreater$ \\
     ~~~~~~~~ + $\textless$element2$\textgreater$Element2$\textless$/element2$\textgreater$) \\
     ~~~~$\textless$/expression1$\textgreater$:
     ~~~~$\textless$element3$\textgreater$Element3$\textless$/element3$\textgreater$ \\
     $\textless$/noeud1$\textgreater$"};

    \node[text width=6cm] (plop) at (-3,-3) {Impression};
    \draw[->] (-2,-2) to [bend right] (-5,-5);

    \draw[->, dashed] (-3,-7) -- (-3,-11);
    \node[text width=6cm] (stdout) at (-2,-12) {Sortie Standard};

    \node[text width=6cm] (plop1) at (10,-3) {Impression};
    \draw[->] (5,-2) to [bend left] (7,-5);

    \draw[->, dashed] (5,-9) -- (5,-11);
    \node[text width=6cm] (stdout) at (7,-12) {Protocole XML};

  \end{tikzpicture}
  \caption{Traduction vers une mise en forme XML\label{xml}}
\end{figure}
}
  \end{frame}

  \begin{tframe}{Processing rules for the code}
  \end{tframe}

  \begin{tframe}{Identifiers}
  \end{tframe}

  \begin{frame}{Code type}
  \end{frame}

  \begin{tframe}{Backend}
  \end{tframe}

\section{Extending Coqdoc : Coq-tex}
  \begin{frame}{Extending Coqdoc : Coq-tex}
    \begin{itemize}[<+->]
      \item Add a new frontend
      \item Process queries
      \item Use the LaTeX backend
      \item Logical constraints on input -> output type
    \end{itemize}
  \end{frame}

\section{Demo}
  \begin{tframe}{Demo}
  \end{tframe}

\section{Remaining work}
  \begin{tframe}{What's left to do}
  \end{tframe}
  \begin{tframe}{New possibilities for Coqdoc}
  \end{tframe}

\section{Conclusion}
  \begin{tframe}{Conclusion}
  \end{tframe}

\end{document}
