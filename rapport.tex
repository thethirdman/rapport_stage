\documentclass[a4paper, 11pt]{report}

\usepackage[utf8]{inputenc}
\usepackage[french]{babel}

\newcommand{\pir}[0]{\xspace\textbf{$\pi r^2$}\xspace}
\newcommand{\coq}[0]{\xspace\textbf{Coq}\xspace}
\newcommand{\coqdoc}[0]{\xspace\textbf{Coqdoc}\xspace}

\title{Rapport de stage de fin de tronc Commun: \\
  Réécriture de Coqdoc}
\author{François Ripault}
\date{FIXME}

\begin{document}
\maketitle
\thispagestyle{empty}

\tableofcontents
\thispagestyle{empty}

\setcounter{page}{0}
\chapter{Introduction}
  Le présent rapport présente les travaux effectués au cours du stage de fin
  de tronc commun à L'epita, réalisé du 3 Septembre 2012 au 11 Janvier 2013.

  Celui s'est déroule au sein de l'INRIA, l'institut national de recherche en
  informatique et en automatique. Celui ci s'est déroule au seins de locaux
  de l'INRIA Rocquencourt, plus particulièrement dans l'antenne Parisienne
  située avenue d'Italie.
  J'ai été intégré au sein de l'équipe de projet \pir qui travaille sur le
  projet \coq, ... %%FIXME

  Le travail principal consistait en la ré-écriture complète du logiciel de
  documentation de la suite d'outils \coq. ... %%FIXME

  \section{Contexte du stage}
  \section{Problématique du stage}
  \section{Motivations et perspectives}
  Le stage offre des perspectives très intéressantes en relation directes
  avec mes objectifs professionnels, tout en mettant en application les
  compétences que j'ai acquises au cours de cycle de tronc commun.
  \subsection{Validation des acquis de tronc commun}
  Le projet proposé met en jeu des compétences propres au métier d'ingénieur.
  En effet, la conception d'un nouveau logiciel divers enjeux en relation directe
  avec les enseignement de l'Epita:
  \begin{itemize}
    \item \textbf{Conception d'un logiciel:} le stage a pour but la ré-écriture à
      neuf d'un logiciel pré-existant. L'objectif est de créer un outil
      s'intégrant bien dans la suite \coq, paliant aux problèmes de l'outil
      précédent, mais également en réutilisant les éléments intéressants pour
      la nouvelle version de \coqdoc.

      La conception d'un nouveau logiciel permet de mettre en application
      les compétences de programmation mais également des aspects
      annexes concernant le développement de projet, telle que
      la documentation du code ou la réalisation de tests unitaires.
    \item \textbf{Gestion de projet:} Ecrire un nouveau logiciel
      apporte également des problématiques concernant la gestion de
      ce projet: il faut choisir un cycle de développement
      approprié permettant dans le temps du stage, de réaliser
      l'application. Il s'agit également de répondre aux attentes
      des utilisateurs concernant le projet, et de faciliter l'intégration
      du logiciel au sein de la suite d'outils \coq.
  \end{itemize}
  \subsection{Introduction au monde de la recherche}
  Ce stage représente également une bonne introduction aux problématiques
  de recherche abordées dans le domaine. En effet, le travail
  au sein de l'équipe de recherche travaillant sur \coq permet d'aborder les
  sujets de recherche liés à la preuve de programme et aux langages
  fonctionnels.

  En plus de cette introduction aux problématiques de recherche dans le domaine,
  ce stage m'offrait une occasion d'élargir mon champ de vision concernant les
  autres domaines de la recherche informatique.

  %%FIXME: parler du CV ?
  \section{Phases de déroulement et introduction du rapport}


\chapter{Présentation de l'entreprise}
  \section{Le secteur d'activité}
  \section{L'entreprise}
  \section{Le service}
  \section{Le positionnement du stage dans les travaux de l'entreprise}
\chapter{Travail effectué}
  \section{Le cahier des charges}
  \section{Compte-rendu d'activité}
  \section{Interprétation et critique des résultats}
\chapter{Conclusion générale}
\chapter{Bibliographie \& glossaire}
\chapter{Annexes}
  \section{Sommaire des Annexes}
  \section{Documentation sur l'entreprise}
  \section{Documentation sur le matériel/les logiciels}
  \section{Résultats bruts}
\end{document}
