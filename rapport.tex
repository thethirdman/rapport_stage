\documentclass[a4paper, 11pt]{report}

\usepackage[utf8]{inputenc}
\usepackage[french]{babel}

\newcommand{\pir}[0]{\textbf{$\pi r^2$}~}
\newcommand{\coq}[0]{\textbf{Coq}\xspace}
\newcommand{\coqdoc}[0]{\textbf{Coqdoc}\xspace}
\newcommand{\yrg}[0]{\xspace\textbf{Yann Régis-Gianas}\xspace}
\newcommand{\epita}[0]{EPITA}

\title{Rapport de stage de fin de tronc Commun: \\
  Réécriture de Coqdoc}
\author{François Ripault}
\date{FIXME}

\begin{document}
\maketitle
\thispagestyle{empty}

\tableofcontents
\thispagestyle{empty}

\setcounter{page}{0}
\chapter{Introduction}
  %% 3 -> 5 pages
  Le présent rapport détaille travaux effectués au cours du stage de fin
  de tronc commun à l'\epita, réalisé du 3 Septembre 2012 au 11 Janvier 2013.

  Le sujet du stage consiste en la refonte complète du logiciel de
  documentation \coqdoc.

  \section{Contexte du stage}
  Ce stage s'est déroulé au sein de l'INRIA, plus particulièrement dans
  l'équipe \pir.

  L'INRIA est l'institut national de recherche en informatique et en
  automatique. C'est un établissement public à caractère scientifique et
  technologique, qui est l'acteur principal de la recherche en informatique en
  France.

  Je fut intégré à l'équipe \pir appartenant au laboratoire Preuves, Programmes et
  Systèmes.

  Le laboratoire \textbf{Preuve, programmes et systèmes} est une unité mixte
  de recherche, rattachée à plusieurs établissements scientifiques (INRIA,
  CNRS, et l'université Paris Diderot), qui regroupe à la fois des
  chercheurs en informatique et en logique mathématique, au sein d'une même
  thématique, celle des langages de programmation, des systèmes distribués,
  et de leurs fondements logiques.

  Au sein de ce laboratoire, l'équipe \pir se concentre sur la recherche
  autour de la correspondance entre preuves et programmes, ayant donné
  naissance à l'assistant de preuves \coq, et le formalisme qui sous-tend
  l'implémentation d'un tel logiciel.

  Ce stage fut effectué sous la direction de \yrg, maître de conférence à
  l'université Paris Diderot.
  %% FIXME

  \section{Problématique du stage}
  L'objectif de ce stage est de refaire le logiciel de documentation de
  l'assistant de preuves \coq, ne répondant plus aux attentes de ses
  utilisateurs.
  En effet, bien que celui-ci remplisse correctement sa tache il souffre d'un
  manque crucial d'extensibilité et s'intègre mal dans la suite d'outils
  fournie avec \coq. Il s'agit donc de refaire entièrement \coqdoc{} afin de
  mieux l'intégrer dans \coq, tout en concevant une architecture qui permette
  de l'étendre facilement par la suite.

  \section{Motivations et perspectives}
  Le stage offre des perspectives très intéressantes en relation directe
  avec mes objectifs professionnels, tout en mettant en application les
  compétences que j'ai acquises au cours de cycle de tronc commun.
  %%% FIXME

  \subsection{Validation des acquis de tronc commun}
  Le projet proposé met en jeu des compétences propres au métier d'ingénieur.
  En effet, la conception d'un nouveau logiciel est en relation directe
  avec les enseignements de l'\epita:
  \begin{itemize}
    \item \textbf{Conception d'un logiciel:} le stage a pour but la réécriture à
      neuf d'un logiciel préexistant: \coqdoc. L'objectif est de créer un outil
      s'intégrant bien dans la suite \coq, palliant aux problèmes de l'outil
      précédent, mais également en réutilisant les éléments intéressants pour
      la nouvelle version de \coqdoc.

      La conception d'un nouveau logiciel permet de mettre en application
      les compétences de programmation mais également des aspects
      annexes concernant le développement de projet, telle que
      la documentation du code ou la réalisation de tests unitaires.
    \item \textbf{Gestion de projet:} Écrire un nouveau logiciel
      apporte également des problématiques concernant la gestion de
      ce projet: il faut choisir un cycle de développement
      approprié permettant dans le temps du stage, de réaliser
      l'application. Il s'agit également de répondre aux attentes
      des utilisateurs concernant le projet, et de faciliter l'intégration
      du logiciel au sein de la suite d'outils \coq.
  \end{itemize}

  \subsection{Introduction au monde de la recherche}
  Ce stage représente également une bonne introduction aux problématiques
  de recherche abordées dans le domaine de l'équipe \pir.

  En effet, bien que le sujet de stage mette en jeu des compétences surtout
  liées au domaine de l'ingénierie, le contexte scientifique du stage me
  permet de découvrir les problématiques de recherche dans le laboratoire PPS
  et plus particulièrement l'équipe de recherche \pir.

  Cela répond directement à mes objectifs professionnels puisque après l'\epita,
  j'ai l'intention de poursuivre dans le monde de la recherche. Ce stage me
  permet donc de valider cet objectif professionnel, tout en tissant des liens
  avec les scientifiques travaillant dans ce domaine.

  %%FIXME: parler du CV ?
  \section{Phases de déroulement et introduction du rapport}
  Ce rapport retrace les différentes phases de déroulement du stage :
  \begin{itemize}
    \item la prise de connaissance des problématiques à résoudre
    \item la conception du logiciel de documentation
    \item l'implémentation de ce logiciel
  \end{itemize}

  Une première partie présentes les différentes structures au sein desquelles
  s'est effectué mon stage, ainsi que l'intégration des réalisation de ce stage
  dans les travaux de l'entreprise. \\
  Dans une deuxième partie, analysons l'existant et définissons la méthodologie
  et les objectifs du stage. \\
  La troisième partie détaillera la phase de réalisation technique.
  Enfin, dans une dernière partie, nous offrons une analyse critique des
  résultats de ce stage.

\chapter{Présentation de l'entreprise}
%% 5 -> 10
  Le stage proposé s'est fait au sein de l'équipe \pir, appartenant au
  laboratoire PPS, regroupant des chercheurs de plusieurs instituts de
  recherche. Cette section présente les instituts de recherche ainsi que le
  laboratoire PPS.
  \section{Le secteur d'activité}
  \section{L'entreprise}
  \section{Le service}
  \section{Le positionnement du stage dans les travaux de l'entreprise}
\chapter{Travail effectué}
%% 40 -> 50
  \section{Le cahier des charges}
    \subsection{But général}
    \subsection{Analyse de l'existant}
    \subsection{Nouveau éléments du logiciel}
  \section{Compte-rendu d'activité}
    \subsection{Conception de l'architecture du logiciel}
    \subsection{Choix techniques de la conception}
    \subsection{Déroulement du développement logiciel}
    \subsection{Résultats du stage}
  \section{Interprétation et critique des résultats}
    \subsection{Version préliminaire de Coqdoc stable}
    \subsection{Extensibilité du logiciel}
    \subsection{Finalisation de Coqdoc}
\chapter{Conclusion générale}
%% 2
\chapter{Bibliographie \& glossaire}
\chapter{Annexes}
  \section{Sommaire des Annexes}
  \section{Documentation sur l'entreprise}
  \section{Documentation sur le matériel/les logiciels}
  \section{Résultats bruts}
\end{document}

