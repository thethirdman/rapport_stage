\documentclass[a4paper, 11pt]{article}
\usepackage[utf8]{inputenc}
\usepackage{graphicx}
\usepackage{parskip}
\usepackage{xspace}
\usepackage{tikz}

\begin{document}

\newcommand{\pir}[0]{\textbf{$\pi r^2$}\xspace}

\begin{titlepage}
\begin{center}
\textsc{\Large Rapport de stage de fin de tronc commun EPITA}

\vfill

\rule{\linewidth}{0.5mm} \\[0.4cm]
{\huge \bfseries Refonte du logiciel Coqdoc}
\rule{\linewidth}{0.5mm} \\[1.5cm]

\begin{minipage}{0.4\textwidth}
\begin{flushleft} \large
\emph{Auteur :}\\
François Ripault \\
\small
\texttt{francois.ripault@epita.fr}
\end{flushleft}
\end{minipage}
\begin{minipage}{0.4\textwidth}
\begin{flushright} \large
\emph{Maître de stage :} \\
Yann Régis-Gianas
\small
\texttt{yrg@pps.univ-paris-diderot.fr}
\end{flushright}
\end{minipage}

\vfill

\fbox{
\begin{minipage}{\textwidth}
\underline{Résumé :}
L'INRIA travaille sur le projet Coq, un assistant de preuve. Au sein de ce
projet, le sujet de ce stage consiste en la refonte du logiciel de
documentation Coqdoc.
L'objectif est de le rendre plus extensible, plus facilement maintenable et mieux l'intégrer au sein
du projet.
\end{minipage}
}

\vfill
Stage effectué à \textbf{L'INRIA} du : 3 Septembre 2012 au 11 Janvier 2013

\begin{figure}[h]
\begin{tabular}{ccccc}
\includegraphics[scale=0.8]{data/inria.jpg} & &
\includegraphics{data/pps.png} & &
\includegraphics[scale=0.8]{data/epita.jpg}
\end{tabular}
\end{figure}

\end{center}
\end{titlepage}


\section{The company}
\subsection{General presentation}
INRIA stands for the National institute of research in computer science and
automatics. It is the main research organism about computer science in France.
The institute has for vocation to undertake applied and fundamental research
in the domain of information and communication sciences and technologies (ICST).

INRIA is a public institute under the leader of the French ministry of research
and the ministry of economy.

The institute has collaborations with many industrial actors, and is also
a start-up creator, which commercialize technologies developed at INRIA.
It also has many international partners, with multinational companies and
foreign universities. It is also active in many international organizations
such as ISO, W3C or IETF.

\section{The service}
\subsection{The proof program and systems laboratory}
The PPS laboratory brings together researchers coming from computer science and
mathematical logic. The laboratory is focused around the idea that logic
and mathematical fields can improve computer science, and that reciprocally,
computer science can advance research in mathematics.

The main research work led by PPS revolves around the Curry-Howard correspondence
between programs and proofs.

\subsection{The $\pi r^2$ research team}

The \pir research team is made of scientists of many scientific instances~: PPS,
INRIA and Paris-Diderot university. Its research topics are the following~:
\begin{itemize}
\item A fundamental research around the correspondence between proofs and programs
\item A theoretical research around the formalism that underlies the Coq proof assistant
\item An implementation field with the development of Coq, especially in the view of Coq as dependently-typed programming language
\end{itemize}
The Coq proof assistant is a software which checks proofs and extracts
certified programs from these proofs. It is used for mathematical proofs, certification
of critical software, and the fundamental research in computer science.

\section{Existing state of affairs}

The Coq project is made of several tools revolving around the Coq compiler, which
is both a proof assistant and a program compiler. One of the tools is Coqdoc,
a documentation software.

Coqdoc allows the programmer to document Coq source files in order to ease the
development of a project, the documentation describing its behavior. This
documentation is included in the source files, making it easy to update.
Coqdoc's role is to translate the source files in order to extract this documentation
and generate a more human readable format, such as a PDF or web-pages. Coqdoc provides
reference manuals and development documentation to the end-user.

\section{Subject of the internship}

The internship subject was to fully rewrite Coqdoc in order to make a better version
of the existing software. Indeed, the previous version has several flaws, motivating a full rewrite :
\begin{itemize}
\item Maintainability difficulty : because of its internal design, Coqdoc is hard
to maintain. One of the goals for this internship is to design a well written software
which is easy to modify and maintaining.
\item Extensibility : the current version of Coqdoc is hard to extend for other
use cases because of its strict design.
\item Redundancy of software elements : Coq's language processing is done in many
  parts inside the Coq project. A full rewrite would allow to reuse thoses
  modules, removing the one included in the previous version of Coqdoc
\end{itemize}

To summarize, the goal of this internship is to make a new documentation tool which is
both simple and extensible.
\section{Timeline of the internship}
The following chart details the different phases of the internship. The software
design revolves around a 3 step processing : a front-end reading the Coq's
source files and translating into an intermediate representation, an
interpretation phase where this representation is translated, and finally a
back-end to generate the documentation. The interpretation phase is based
around an interaction protocol with the compiler, which was very complex to
develop.
\newcommand{\start}[3]
{
  \node[text width=3cm] at (0 - 1,#2) {#1};
  \node at (0,#2) [circle, fill=black] {};
  \node[text width=10cm] at (0 + 6, #2) {#3};
}

\newcommand{\pt}[3]{
  \node[text width=3cm] at (0 - 1,-#2) {#1};
  \node at (0,-#2) [circle, fill=black] {};
  \node[text width=10cm] at (0 + 6, -#2) {#3};

  \draw (0, -#2) -- (0,0);
}

\begin{tikzpicture}
\start{Week 1 - 2}{0}{
Analysis of the existing state of affairs, conception of the software design};
\pt{Week 2 - 4}{2}{Writing of the front-end, language processing into an
intermediate representation};
\pt{Week 4 - 6}{4}{Refinement of the intermediate generic representation};
\pt{Week 6 - 10}{6}{Development of the interaction protocol};
\pt{Week 10 - 12}{8}{Development of the evaluation module processing the
 intermediate representation};
\pt{Week 12 - 14}{10}{Development of the back-end}
\pt{Week 14 - 16}{12}{Finalization of the software and bug-fixes}


\end{tikzpicture}
\clearpage
\section{Internship feedback}
This internship was very enjoyable for many reasons. I learned a lot both
technically and theoretically, on the aspects of functional programming and
software development. The other works done in the laboratory are very interesting,
and opened my vision on other research and work topics in the domain of computer
science

I met a lot of interesting people who provided me with a lot of insight on
my future career and professional orientation after EPITA.

\section{Skills acquired}
The following list summarizes the skills I acquired or improved during this internship.
\begin{itemize}
\item Learn a lot about software design
\item Development of complex applications
\item Improvements on the Ocaml programming language
\item Better understanding of language processing
\end{itemize}
\section{Conclusion}

This internship was a success on many aspects, and I enjoyed it. The team
was very welcoming and I appreciated a lot the atmosphere in the laboratory.
I improved a lot of skills during the time spent at INRIA, and discovered the
research subjects of the laboratory in which I worked.
\end{document}
