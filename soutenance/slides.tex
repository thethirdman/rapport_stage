\documentclass[compress]{beamer}

\usepackage[utf8]{inputenc}
\usepackage[T1]{fontenc}
\usepackage[french]{babel}
\usetheme{Amsterdam}

\usepackage{multicol}
\usepackage{listings}
\usepackage{color}

\usepackage{tikz}
\usetikzlibrary{shapes,arrows}

\lstset{language=caml, frame=single, basicstyle=\ttfamily\scriptsize, commentstyle=\color{dkgreen}}

\newcommand{\email}[0]{francois.ripault@epita.fr}

\definecolor{dkgreen}{rgb}{0,0.6,0}

\defbeamertemplate*{title page}{customized}[1][]
{
  \begin{center}
  \usebeamerfont{title}\inserttitle\par
    \usebeamerfont{subtitle}\usebeamercolor[fg]{subtitle}\insertsubtitle\par
    \bigskip
    \usebeamerfont{author}\insertauthor\par
    \scriptsize \texttt{\email} \par
    \usebeamerfont{institute}\insertinstitute\par
    \usebeamerfont{date}\insertdate\par
    %\usebeamercolor[fg]{titlegraphic}\inserttitlegraphic
    \end{center} }

\newenvironment{tframe}[1]{
  \subsection{#1}
  \begin{frame}{#1}
  }{
  \end{frame}
  }

%% FIXME: title
\title{Soutenance de stage de fin de tronc commun : \\ Refonte du logiciel Coqdoc}

\author{François Ripault}

\begin{document}
\begin{frame}
  \titlepage
\end{frame}

\begin{frame}{Introduction}
  Motivations de ce stage

  \vfill

  \begin{columns}[2]
    \begin{column}{0.5\textwidth}
    Aspect ingénierie :
  \begin{itemize}
    \item Validation des acquis
    \item Acquisition de nouvelles compétences
    \item Production d'un logiciel
  \end{itemize}
    \end{column}
    \begin{column}{0.5\textwidth}
      Aspect recherche :
  \begin{itemize}
    \item Introduction aux sujets de recherche
      de PPS
    \item Interaction avec des scientifiques
  \end{itemize}
    \end{column}
  \end{columns}
\end{frame}

\begin{frame}
\tableofcontents
\end{frame}

\section{Presentation de l'entreprise}
\begin{tframe}{L'INRIA}
  Présentation de l'entreprise :
  \begin{itemize}
    \item Institut national de recherche en informatique et en automatique
    \item Activité scientifique centrée sur l'informatique, et ses interactions
      avec d'autres sciences.
    \item Collaboration avec de nombreux partenaires industriels et académiques
    \item Créateur de start-ups
  \end{itemize}
\end{tframe}
\begin{tframe}{Le laboratoire PPS}
  Présentation du service
  \begin{itemize}
    \item Laboratoire Preuves, Programmes et Systèmes
    \item Equipe multi-disciplinaire
    \item Liens entre les mathématiques et l'informatique
  \end{itemize}
\end{tframe}

\section{Enseignements du stage}

\begin{tframe}{Validation des acquis}
\end{tframe}
\begin{tframe}{Obtention de nouvelles compétences}
\end{tframe}
\begin{tframe}{Réalisation complète d'un logiciel}
\end{tframe}
\begin{tframe}{Interactions avec le monde de la science}
\end{tframe}

\begin{frame}{Conclusion}
\end{frame}

\end{document}
