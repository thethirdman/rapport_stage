\documentclass[compress]{beamer}

\usepackage[utf8]{inputenc}
\usepackage[T1]{fontenc}
\usepackage[french]{babel}
\usetheme{Amsterdam}

\usepackage{multicol}
\usepackage{listings}
\usepackage{color}

\usepackage{tikz}
\usetikzlibrary{shapes,arrows}

\lstset{language=caml, frame=single, basicstyle=\ttfamily\scriptsize, commentstyle=\color{dkgreen}}

\newcommand{\email}[0]{francois.ripault@epita.fr}

\definecolor{dkgreen}{rgb}{0,0.6,0}

\defbeamertemplate*{title page}{customized}[1][]
{
  \begin{center}
  \usebeamerfont{title}\inserttitle\par
    \usebeamerfont{subtitle}\usebeamercolor[fg]{subtitle}\insertsubtitle\par
    \bigskip
    \usebeamerfont{author}\insertauthor\par
    \scriptsize \texttt{\email} \par
    \usebeamerfont{institute}\insertinstitute\par
    \usebeamerfont{date}\insertdate\par
    \vfill
    \begin{figure}[h]
      \begin{tabular}{ccccc}
        \includegraphics[scale=0.4]{../data/inria.jpg} & &
        \includegraphics[scale=0.6]{../data/pps.png} & &
        \includegraphics[scale=0.4]{../data/epita.jpg}
      \end{tabular}
    \end{figure}
    %\usebeamercolor[fg]{titlegraphic}\inserttitlegraphic
    \end{center} }

\newenvironment{tframe}[1]{
  \subsection{#1}
  \begin{frame}{#1}
  }{
  \end{frame}
  }

%% FIXME: title
\title{Soutenance de stage de fin de tronc commun : \\ Refonte du logiciel Coqdoc}

\author{François Ripault}

\begin{document}
\begin{frame}
  \titlepage
\end{frame}

\begin{frame}{Introduction}
  \underline{Motivations de ce stage :}

  \vfill

  \begin{columns}[2]
    \begin{column}{0.5\textwidth}
    Aspect ingénierie :
  \begin{itemize}
    \item Validation des acquis
    \item Acquisition de nouvelles compétences
    \item Production d'un logiciel
  \end{itemize}
    \end{column}
    \begin{column}{0.5\textwidth}
      Aspect recherche :
  \begin{itemize}
    \item Introduction aux sujets de recherche
      de PPS
    \item Interaction avec des scientifiques
  \end{itemize}
    \end{column}
  \end{columns}
\end{frame}

\begin{frame}
\tableofcontents
\end{frame}

\section{Presentation de l'entreprise}
\begin{tframe}{L'INRIA}
  Présentation de l'entreprise :
  \begin{itemize}
    \item Institut national de recherche en informatique et en automatique
    \item Activité scientifique centrée sur l'informatique, et ses interactions
      avec d'autres sciences.
    \item Collaboration avec de nombreux partenaires industriels et académiques
    \item Créateur de start-ups
  \end{itemize}
  \begin{figure}
        \includegraphics[scale=0.4]{../data/inria.jpg}
  \end{figure}
\end{tframe}
\begin{tframe}{Le laboratoire PPS}
  Présentation du service
  \begin{itemize}
    \item Laboratoire Preuves, Programmes et Systèmes
    \item Equipe multi-disciplinaire
    \item Liens entre les mathématiques et l'informatique
  \end{itemize}

  \medskip

  Le Projet Coq
  \begin{itemize}
    \item Assistant de preuves
    \item Certifications de programmes, validation de preuves, recherche
      fondamentale en informatique
    \item Possède un logiciel de documentation : Coqdoc
  \end{itemize}
  \begin{figure}
        \includegraphics[scale=0.6]{../data/pps.png}
  \end{figure}
\end{tframe}

\section{Enseignements du stage}

\begin{tframe}{Validation des compétences}
      Compétences techniques :
      \begin{itemize}
        \item Perfectionnement en Ocaml
        \item Traitement du langage
      \end{itemize}

      \medskip

      Compétences de ``gestion de projet'' :
      \begin{itemize}
        \item Commentaire de code
        \item Réalisation d'une base de test
        \item Intégration du logiciel réalisé
      \end{itemize}
\end{tframe}

\begin{tframe}{Conception Logicielle}
      Réécriture entière de Coqdoc :
      \begin{itemize}
        \item Conception de l'architecture
        \item Création d'un logiciel extensible
        \item Utilisation de l'existant
        \item Intégration du logiciel
      \end{itemize}
\end{tframe}

\begin{tframe}{Aspect scientifique}
      Monde de la science :
      \begin{itemize}
        \item Meilleure compréhension des sujets de recherche dans PPS
        \item Conaissances dans le domaine de la programmation fonctionnelle
        \item Ouverture d'esprit sur d'autres sujets de recherche
      \end{itemize}
\end{tframe}

\begin{frame}{Conclusion}
  \begin{itemize}
    \item Stage enrichissant
    \item Sujet très lié à l'aspect ingénierie
    \item Fortement ancré dans le monde de la recherche
  \end{itemize}
\end{frame}

\end{document}
